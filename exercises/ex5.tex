\newpage
\subsection{Aufgabe 5: ER-Diagramm}
\label{sec:Aufgabe5}
\textbf{Die Aufgaben (a) - (c) waren in einem Fließtext formuliert}.
\begin{enumerate}[label=\alph*)]
    \item Erstellen Sie zu folgendem Sachverhalt ein ER-Diagramm \textbf{ohne} Verwendung von Beziehungen.
    \\[0.5cm]
    Ein Auto hat:
    \begin{enumerate}[label=\arabic*.]
        \item Eindeutige Fahrgestellnummer.
        \item Ausstattung. (Klimaanlage, Sitzheizung, etc.)
        \item Verbrauch, welcher sich aus dem Gewicht und dem Motor (klein, mittel, groß) zusammensetzt.
        \item Hersteller, zu dem der Ort, die Straße und die Plz bekannt sind.
    \end{enumerate}
    
    \item Erstellen Sie zu folgendem Sachverhalt ein ER-Diagramm \textbf{mit} Verwendung von Beziehungen.
    \\[0.5cm]
    Ein Mieter kann mehrere Autos mieten. Ein Auto kann von mehreren Nutzern gemietet werden.
    \\
    Mietet ein Mieter ein Auto, so wird dabei der Start- und der Endzeitpunkt festgelegt.
    \\
    Ein Auto kommt mehrmals zur Reinigung. Jede Reinigung ist genau einem Fahrzeug zugeordnet.
    \\[0.2cm]
    Eine Reinigung hat:
    \begin{enumerate}[label=\arabic*.]
        \item Datum.
        \item Hallenbereich.
        \item Dieses Wertepaar ist alleine nicht eindeutig.
    \end{enumerate}
    Ein Mieter hat:
    \begin{enumerate}[label=\arabic*.]
        \item Email Adresse.
        \item Vor- und Nachname.
        \item Diese Werte zusammen sind einem Mieter eindeutig zuzuordnen.
    \end{enumerate}

    \item
    Erstellen Sie zu folgendem Sachverhalt ein ER-Diagramm mit einer Min-Max Kardinalität
    \\[0.5cm]
    Ein Nutzer muss entweder Anbieter oder Mieter sein. Er kann allerdings auch beide Rollen einnehmen.

    \item Notieren Sie die Lösung aus Aufgabe (c) in der Relationsschreibweise \textit{attribut(\underline{wert1},...))}.
\end{enumerate}
