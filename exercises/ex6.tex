\newpage
\subsection{Aufgabe 6: XML}
\label{sec:Aufgabe6}
\begin{enumerate}[label=\alph*)]
    \item Gegeben ist folgendes XML. Wie sieht das zugehörige DTD aus? (Auch hier kommt es zu Abweichungen aus der Klausur)
    \begin{minted}{XML}
        <buch>
            <inhalt>
                <prolog iid='i1'>
                    Hier stand etwas...
                </prolog>
                <hauptteil iid='i2'>
                    Hier stand etwas...
                </hauptteil>
                <schluss iid='i3'>
                    Hier stand etwas...
                </schluss>
            </inhalt>
            <kapitel kid='k1' iid='i1'>
                <einleitung seitenanzahl=5>
                    Hier stand etwas...
                </einleitung>
                <abschnitt seitenanzahl=12>
                    Hier stand etwas...
                </abschnitt>
                <abschnitt seitenanzahl=3>
                    Hier stand etwas...
                </abschnitt>
            </kapitel>
            <kapitel kid='k2'>
                <abschnitt seitenanzahl=2>
                    Hier stand etwas...
                </abschnitt>
                <abschnitt seitenanzahl=13>
                    Hier stand etwas...
                </abschnitt>
            </kapitel>
        </buch>
    \end{minted}
    
    \item Wann spricht man von wohlgeformten XML?
    \item Wie lauten die XPath Befehle für folgende Abfragen?
    \begin{enumerate}[label=\arabic*.]
        \item Geben Sie den Text aus der Einleitung des ersten Kapitels an.
        \item Aus wie vielen Seiten besteht das erste Kapitel?
        \item Geben Sie das letzte Kapitel aus.
    \end{enumerate}
\end{enumerate}
